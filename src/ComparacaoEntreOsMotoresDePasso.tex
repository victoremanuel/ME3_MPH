O motor de passo híbrido em comparação aos outros motores de passo tem largura de passo menor e maior torque para menores tamanhos. Além de ter a característica de conseguir manter a posição do rotor caso falte energia de alimentação, pelo fato de ele ter no rotor um campo permanente criado pelo imã, produzindo um pequeno "torque de retenção" capaz de manter o eixo na posição parada estável. Na Tabela \ref{my-label} é mostrado a comparação entre os tipos de motores de passo. \cite{SteppingBook}

\begin{table*}[!]
	\centering
	\caption{Comparação entre os motores de passo VRM, PMM e HSM (Microchip WebSeminars, 2012)}
	\label{my-label}
	\begin{tabular}{|c|c|c|c|}
		\hline
		\textit{Característica} & \textbf{PMM} & \textbf{VRM} & \textbf{HSM} \\ \hline
		\textit{\begin{tabular}[c]{@{}c@{}}Custo de \\ produção\end{tabular}} & Barato & Moderado & Muito Caro \\ \hline
		\textit{Design} & \begin{tabular}[c]{@{}c@{}}Moderadamente \\ Complexo\end{tabular} & Simples & Complexo \\ \hline
		\textit{Resolução} & 30º-3º/passo & \multicolumn{2}{c|}{1,8º/passo ou menor} \\ \hline
		\textit{Ruído} & Não há & \begin{tabular}[c]{@{}c@{}}Há ruido independente\\ do tipo de excitação\end{tabular} & Não há \\ \hline
		\textit{Passo} & \begin{tabular}[c]{@{}c@{}}Passo cheio, meio passo\\ e micro-passo\end{tabular} & \begin{tabular}[c]{@{}c@{}}Normalmente opera em\\ passo cheio\end{tabular} & \begin{tabular}[c]{@{}c@{}}Passo cheio, meio passo\\ e micro-passo\end{tabular} \\ \hline
	\end{tabular}
\end{table*}