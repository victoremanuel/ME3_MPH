Os motores de passo, inclusive o de passo híbrido, são amplamente utilizados em sistemas de controle digital, motores de alimentação de papel e posicionamento da cabeça em unidade de discos, entre outras. Por não terem enrolamentos no rotor, eles têm boa densidade de potência, pois não há necessidade de grande dissipação de calor do rotor, e também têm grande aplicação onde é necessário realizar movimentos precisos e ter conhecimento da posição do rotor sem necessidade de realimentação com sensores. O motor de passo híbrido ainda tem melhores características de conjugado, velocidade e precisão nos passos. \cite{Fitz}

Algumas das principais aplicações dos motores de passo híbridos são: \cite{aplciacoes}

\begin{itemize}
	\item Medicina e automação laboratorial: 
	\subitem bombas peristálticas e seringas,
	\subitem análises,
	\subitem \textit{scanners} ótico,
	\subitem máquinas de dispersão para farmácias,
	\subitem imagens dentais,
	\subitem sistemas de manuseio e movimento de fluidos.
	\item Fábricas têxtil
	\subitem sistemas de monitoramento de fios,
	\subitem máquinas de acolchoamento de tapetes,
	\subitem em máquinas de fiação.
	\item Automação fabril
	\subitem equipamentos de semicondutores,
	\subitem carcaça de eletrônicos,
	\subitem equipamento de embalagem,
	\subitem transportadoras.
	\item Telecomunicações
	\subitem antenas de telefonia celular
	\subitem GPS,
	\subitem posicionamento de antenas,
	\subitem posicionamento de radares.
	\item Outros
	\subitem automatização de impressoras e copiadoras,
	\subitem bilhetagem,
	\subitem automação de escritórios,
	\subitem gravação.
\end{itemize}