Os motores de passo, inclusive o de passo híbrido, são amplamente utilizados em sistemas de controle digital, motores de alimentação de papel e posicionamento da cabeça em unidade de discos, entre outras. Por não terem enrolamentos no rotor, eles têm boa densidade de potência, pois não há necessidade de grande dissipação de calor do rotor, e também têm grande aplicação em aplicações onde é necessário realizar movimentos precisos e ter conhecimento da posição do rotor sem necessidade de realimentação com sensores. O motor de passo híbrido ainda tem melhores características de conjugado, velocidade e precisão nos passos. \cite{Fitz}