\usepackage{float}

\titulo{Introdução ao Motor de Passo Híbrido -- Princípio de funcionamento, características construtivas, modelo matemático e aplicações}

\title{Introduction to Hybrid Stepper Motor -- Principle of operation, construction characteristics, mathematical model and applications}

\author{Callebe S. Barbosa$^{1}$, Rafael C. Bonotto$^{2}$, Raphael H. Machado$^{3}$, Victor Emanuel S. Barbosa$^{4}$\\
	\normalsize e-mail: $^{1}$callebe@alunos.utfpr.edu.br, $^{2}$bonotto@alunos.utfpr.edu.br, 
	$^{3}$raphaelhenrique26@gmail.com, $^{4}$victorbarbosa@alunos.utfpr.edu.br
}

\begin{document}

\maketitle

\begin{resumo}
	O objetivo deste documento é descrever o princípio de funcionamento, características construtivas, modelo matemático da motor de passo híbrido, além de mostrar algumas aplicações.
\end{resumo}

\begin{palavraschave}
	Elétrica, Máquinas, Máquinas elétricas, Motor de passo híbrido, Motor.
\end{palavraschave}

\englishtitle

\begin{abstract}
	The purpose of this document is to describe the principle of operation, construction characteristics, mathematical model of hybrid stepper motor, and show some applications.
\end{abstract}

\begin{keywords}
	Electrical, Machinery, Electrical equipment, Hybrid stepper motor, Motor.
\end{keywords}

%\newpage
\section*{NOMENCLATURAS E SIGLAS}

\symbolnomenclature{HSM}{\textit{Hybrid Stepper Motor}, Motor de passo híbrido}
\symbolnomenclature{VRM}{\textit{Variable Reluctance Motor}, Motor de relutância variável}
\symbolnomenclature{PMM}{\textit{Permanent Magnet Motor}, Motor de imã permanente}
