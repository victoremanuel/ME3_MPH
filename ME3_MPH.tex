\documentclass[portugues]{sobraep}

\titulo{MOTOR DE PASSO HÍBRIDO}

\title{HYBRID STEPPER MOTOR}


\author{Callebe S. Barbosa$^{1}$, Rafael C. Bonotto$^{2}$, Raphael H. Machado$^{3}$, Victor Emanuel S. Barbosa$^{4}$\\
	\normalsize e-mail: $^{4}$victorbarbosa@alunos.utfpr.edu.br, $^{2}$bonotto@alunos.utfpr.edu.br, @alunos.utfpr.edu.br, @alunos.utfpr.edu.br
}

\begin{document}

\maketitle

\begin{resumo}
	O objetivo deste documento é descrever o princípio de funcionamento, características construtivas, modelo matemático da motor de passo híbrido, além de mostrar algumas aplicações.
\end{resumo}

\begin{palavraschave }
	Elétrica, Máquinas, Máquinas elétricas, Motor de passo híbrido, Motor.
\end{palavraschave }

\englishtitle

\begin{abstract}
	The purpose of this document is to describe the principle of operation, construction characteristics, mathematical model of hybrid stepper motor, and show some applications.
\end{abstract}

\begin{keywords}
	Electrical, Machinery, Electrical equipment, Hybrid stepper motor, Motor.
\end{keywords}

%\newpage
\section*{NOMENCLATURA}

\symbolnomenclature{HSM}{\textit{Hybrid Stepper Motor}, Motor de passo híbrido}

%~~~~~~~~~~~~~~~~~~~~~~~~~~~~~~~~~~~~~~~~
% Seções
%~~~~~~~~~~~~~~~~~~~~~~~~~~~~~~~~~~~~~~~~

% Introdução
\section{INTRODUÇÃO}
	
	Os motores de passo híbrido, ou \textit{Hybrid Stepper Motor} (da sigla \textbf{HSM}) como são citados na literatura, são comumente utilizados em aplicações que demandam de alta precisão em motores de corrente contínua \cite{ieeeRusso}.
	
	O HSM, como será referido neste trabalho, combina as vantagens do motor de relutância variável (\textit{Variable Reluctance Motor}, da sigla \textbf{VRM}) e do motor de imã permanente (\textit{Permanent Magnet Motor}, da sigla \textbf{PMM}). Como em outros motores elétricos, o motor de passo híbrido é composto por um estator e um rotor com certas características que caracterizam-o como um motor de passo híbrido.
	
	O estator é composto por um dado número de polos magnéticos relacionados com os enrolamentos colocados nele, com um número específico de dentes do estator. %explicar melhor
	
	O rotor é composto por núcleos magnéticos, com quantidade dependente do número de fases de acionamento do motor, com diferentes quantidades de dentes que são separados por imãs permanentes. Essa quantidade %terminar

\section{Princípio de funcionamento}
Como explanado anteriormente, o motor de passo híbrido congrega características do motor de passo de ímã permanente e o motor de passo de relutância variável. Para entender como o motor de passo híbrido funciona, primeiramente irá ser mostrado brevemente o princípio de funcionamento destes dois outros tipos de motores de passo ora citados, bem como o que são motores de passo, por fim será explicado o funcionamento do motor de passo híbrido em função como um motor aperfeiçoado dos outros dois motores de passo, a fim da explicação se tornar mais familiarizada para o leitor.  

	\subsection{O que é motor de passo?}
	São máquinas elétricas que consistem em um estator com enrolamentos de excitação e um rotor magnético com saliências. Neles o conjugado é produzido pela tendência do rotor a se alinhar com a onda de fluxo produzida pelo estator, de modo a maximizar os fluxos concatenados que resultam da aplicação de uma dada corrente no estator. No motor de passo as fases dos enrolamentos do estator são excitadas sequencialmente, fazendo o rotor girar na forma de uma sequência de passos, com ângulos definidos a cada passo, devido à tendência de alinhamento do rotor com a onda de fluxo do estator. As principais características dos motores de passo são: 
	
	\begin{itemize}
		\item \textbf{Inexistência de escovas:} não necessitam de escovas, reduzindo a maioria das falhas comumente encontradas nos outros motores elétricos, como faiscamento e perdas ôhmicas no rotor e escovas.
		\item \textbf{Independência da carga:} giram com uma dada velocidade independentemente da carga, desde que tal carga não exceda as característica de torque do motor.
		\item \textbf{Menos sensores} se movem com incrementos quantificáveis, desde que com torque especificado, podendo ter conhecimento da posição do eixo.
		\item \textbf{Posição de repouso:} é possível manter o eixo estacionário, desde que seu torque seja respeitado. 
	\end{itemize}
	
	\subsection{Quais os tipos construtivos de motores de passo?}
	
	\subsection{Funcionamento do motor de passo híbrido}

\section{Características construtivas}

\section{Modelo matemático}

\section{Exemplos de aplicações}
Os motores de passo, inclusive o de passo híbrido, são amplamente utilizados em sistemas de controle digital, motores de alimentação de papel e posicionamento da cabeça em unidade de discos, entre outras. Por não terem enrolamentos no rotor, eles têm boa densidade de potência, pois não há necessidade de grande dissipação de calor do rotor, e também têm grande aplicação em aplicações onde é necessário realizar movimentos precisos e ter conhecimento da posição do rotor sem necessidade de realimentação com sensores. O motor de passo híbrido ainda tem melhores características de conjugado, velocidade e precisão nos passos.

\section{CONCLUSÕES}



\bibliographystyle{bib_sobraep}
\bibliography{referencias_sobraep}

\end{document}